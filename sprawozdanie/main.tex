\documentclass[12pt,a4paper]{article}

\usepackage[polish]{babel}
\usepackage[utf8]{inputenc}
\usepackage[T1]{fontenc}
\usepackage{lmodern}
\usepackage{geometry}
\geometry{a4paper, margin=1in}

\usepackage{graphicx}

\title{Sprawozdanie z miniprojektu 3}
\author{Mateusz Kamieniecki}

\begin{document}
\maketitle
\begin{tabular}{|p{8cm}|p{6cm}|}
	\hline
	\textbf{SPRAWOZDANIE}        & \textbf{Data wykonania: 29.11.2025} \\
	\hline
	\textbf{Tytuł miniprojektu:} & \textit{Grafy}                      \\
	\hline
	\textbf{Wykonał:}            & \textit{Mateusz Kamieniecki}        \\
	\hline
	\textbf{Sprawdził:}          & \textit{dr inż. Konrad Markowski}   \\
	\hline
\end{tabular}
\tableofcontents

\section{Cel projektu}
Celem projektu było stworzenie losowego spójnego grafu, przedstawienie go w
postaci macierzy sąsiedztwa, macierzy incydencji oraz listy sąsiedzwtwa.
Program miał wypisywać wyrysowany graf, listę lub macierz odpowiadającą grafu,
stopnie wierzchołków oraz podać nam wszystkie łuki i multiłuki w grafie. Ja
osobiście uznałem, że łuki i multiłuki będą podane podczas rysowania grafu.

\section{Rozwiązanie problemu}
Aby rozwiązać problem musimy najpierw stworzyć spójny graf. W tym celu musimy
swtorzyć minimalne drzewo rozpinające do którego później będziemy dodawać
resztę krawędzie.

\section{Szczegóły implementacji}

\section{Sposób wykonywania programu}

\section{Wnioski i spostrzeżenia}

\end{document}

