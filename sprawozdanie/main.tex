\documentclass[12pt,a4paper]{article}

\usepackage[polish]{babel}
\usepackage[utf8]{inputenc}
\usepackage[T1]{fontenc}
\usepackage{lmodern}
\usepackage{geometry}
\geometry{a4paper, margin=1in}

\usepackage{graphicx}
\usepackage{listings}

\title{Sprawozdanie z miniprojektu 3}
\author{Mateusz Kamieniecki}

\lstset{
    inputencoding=utf8,
    breakatwhitespace=true, 
    breaklines=true,       
    extendedchars=true,
    literate=%
    {ą}{{\k{a}}}1
    {ć}{{\'c}}1
    {ę}{{\k{e}}}1
    {ł}{{\l{}}}1
    {ń}{{\'n}}1
    {ó}{{\'o}}1
    {ś}{{\'s}}1
    {ź}{{\'z}}1
    {ż}{{\.z}}1
}

\begin{document}
\maketitle
\begin{tabular}{|p{8cm}|p{6cm}|}
	\hline
	\textbf{SPRAWOZDANIE}        & \textbf{Data wykonania: 29.11.2025} \\
	\hline
	\textbf{Tytuł miniprojektu:} & \textit{Grafy}                      \\
	\hline
	\textbf{Wykonał:}            & \textit{Mateusz Kamieniecki}        \\
	\hline
	\textbf{Sprawdził:}          & \textit{dr inż. Konrad Markowski}   \\
	\hline
\end{tabular}
\tableofcontents

\section{Cel projektu}
Celem projektu było stworzenie losowego spójnego grafu, przedstawienie go w
postaci macierzy sąsiedztwa, macierzy incydencji oraz listy sąsiedzwtwa.
Program miał wypisywać wyrysowany graf, listę lub macierz odpowiadającą grafu,
stopnie wierzchołków oraz podać nam wszystkie łuki i multiłuki w grafie. Ja
osobiście uznałem, że łuki i multiłuki będą podane podczas rysowania grafu.

\section{Rozwiązanie problemu}
Aby rozwiązać problem musimy najpierw stworzyć spójny graf. W tym celu musimy
swtorzyć minimalne drzewo rozpinające do którego później będziemy dodawać
resztę krawędzie.

\subsection{Tworzenie losowego grafu spójnego}
Minimalne drzewo rozpinające tworzymy poprzez losowe łączenie wierzchołków,
które jeszcze nie miały połączenia. Gdy mamy już połączone wszystkie
wierzchołki, mamy gotowe minimalne drzewo rozpinające.

Gdy mamy już minimalne drzewo rozpinające, możemy dodać resztę krawędzi.
Robimy to poprzez losowe wybieranie dwóch wierzchołków i dodawanie między nimi
krawędzi. Krawędzie teraz mogą się powtarzać, więc możemy mieć multiłuki.

Dla każdej metody reprezentacji grafu (macierz sąsiedztwa, macierz incydencji,
lista sąsiedztwa) będziemy korzystać z macierzy sądziedztwa do tworzenia grafu,
ponieważ ona jest najprostsza do zrobienia.

\subsection{Macierz sąsiedztwa}

\subsubsection{Rysowanie grafu}
Gdy mamy wygenerowany graf w postaci macierzy sąsiedztwa, możemy go narysować
iterując się przez tablicę i wypisać krawędź jak napotkamy niezerową wartość w
tablicy. Liczba w danej komórce będzie reprezentować liczbę krawędzi między
dwoma wierzchołkami.

\subsubsection{Stopnie wierzchołków}
Aby obliczyć stopnie wierzchołków, iterujemy się przez macierz sąsiedztwa i dla
każdej niezerowej wartości w komórce dodajemy do stopnia wierzchołka wartość
z komórki odpowiednio do stopnia wejśćia z danego wierzchołka oraz stopnia wyjścia danego wierzchołka.

\subsection{Macierz incydencji}

\subsubsection{Rysowanie grafu}
Aby narysować graf z macierzy incydencji,
iterujemy się przez kolumny macierzy. Dla każdej kolumny sprawdzamy, które
wiersze mają wartość 1 (krawędź wychodząca) oraz -1 (krawędź wchodząca). Gdy
zliczyliśmy wszystkie krawędzie, wypisujemy je.

\subsubsection{Stopnie wierzchołków}
Aby obliczyć stopnie wierzchołków z
macierzy incydencji, iterujemy się przez wiersze macierzy. Dla każdego wiersza
zliczamy liczbę wystąpień 1 (stopień wyjścia) oraz -1 (stopień wejścia).

\subsection{Lista sąsiedztwa}

\subsubsection{Rysowanie grafu}
Aby narysować graf z listy sąsiedztwa, iterujemy się przez każdy wierzchołek i
wypisujemy wszystkie krawędzie wychodzące z danego wierzchołka oraz ewentualnie
ilość krawędzi.

\subsubsection{Stopnie wierzchołków}
Aby obliczyć stopnie wierzchołków z listy
sąsiedztwa, iterujemy się przez każdy wierzchołek. Stopień wejścia dla danego
wierzchołka obliczamy poprzez zliczanie ile razy dany wierzchołek pojawia, a
stopień wyjścia to po prostu długość listy sąsiedztwa danego wierzchołka.

\section{Szczegóły implementacji}
\begin{lstlisting}[language=C] 
\end{lstlisting}

\subsection{Tworzenie losowego grafu spójnego}


\section{Sposób wykonywania programu}

\section{Wnioski i spostrzeżenia}

\end{document}

